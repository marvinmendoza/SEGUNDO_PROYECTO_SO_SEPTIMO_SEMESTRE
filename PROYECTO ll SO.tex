\documentclass[10pt,a4paper]{article}
\usepackage[utf8]{inputenc}
\usepackage{amsmath}
\usepackage{amsfonts}
\usepackage{amssymb}

\author{MARVIN MENDOZA, EDER RODRIGUEZ}
\title{PROYECTO ll SISTEMAS OPERATIVOS}



\begin{document}
\title{RECURSOS COMPARTIDOS EN SERVIDOR WINDOWS A USUARIOS WINDOWS, UBUNTU Y ANDROID}
\maketitle
Para realizar el Proyecto ll correspondiente al curso de Sistemas Operativos 2 se utilizó windows server 2019




\section{Presentación}
Como segundo proyecto se ha solicitado realizar el uso compartido de una carpeta con varios usuarios entre ellos linux, windos y android.
Los servivios fueron levantados en un servidor de Windows server 2019.
El dominio que se ha utilizado es sistemaso,local

\section{COMO CONECTARME}
El método de conexión es por medio de usuario creados por el administrador del servidor, en nuetro servidor se han creado departamentos los cuales estos departamentos se van incorporado los trabajadores o usuarios, creado el usuario se le asigna el la contraseña y su respectivo correo corporativo.

Al ingresar desde su computadora ingresa el usuario proporcionado y contraseña y al momento de ingresar ya cuenta con acceso a la carpeta que se esta compartiendo entre los usuarios.


\section{COMO ACCEDER DESDE UN SISTEMA ANDROID}
Para poder conectarme a la carpeta compartida del servidor de windows server 2019 descargamos la aplicación Cx Explorador de archivos. Cuando ya este instalada ingresamos a la opción nueva ubicación, seguidamente en REMOTO, seleccionamos Red local, seleccionamos el servidor e ingresamos la contraseña y usuario que ya nos han proporcionado, y de esta forma podemos consulutar la o las carpetas compartidas entre varios usuarios.

\subsection{LINK DE VIDEO EXPLICANDO COMO FUNCIONA, Y REPOSITORIO EN GITHUB}
https://drive.google.com/drive/folders/1dOSgG9GNziXICpxyHCvDcqVjghHDxHZG?usp=sharing



\end{document}